\chapter{ODOO PLM 在協同設計中的應用}
在這章節中我們將使用1D系統的鋼球平衡台來展示ODOO中產品生命週期(PLM)的功能,首先我們來到ODOO主介面並選取產品生命週期(PLM)。

\begin{figure}[h]
\centering
\includegraphics[width=0.8\textwidth]{../images/圖5.1.png}
\caption{X}
\end{figure}

進到這個頁面之後選擇主資料中的產品。

\begin{figure}[h]
\centering
\includegraphics[width=0.8\textwidth]{../images/圖5.2.png}
\caption{圖5.2}
\end{figure}

接下來按下新增(圖5.3)進入到圖5.4

\begin{figure}[h]
\centering
\includegraphics[width=0.16\textwidth]{../images/圖5.3.png}
\caption{圖5.3}
\end{figure}

\begin{figure}[h]
\centering
\includegraphics[width=0.8\textwidth]{../images/圖5.4.png}
\caption{圖5.4}
\end{figure}

我們以鋼球平衡台作為範例

\begin{figure}[h]
\centering
\includegraphics[width=0.8\textwidth]{../images/圖5.5.png}
\caption{圖5.5}
\end{figure}

選擇物料清單來新增鋼球平衡台所需的零件。

\begin{figure}[h]
\centering
\includegraphics[width=0.5\textwidth]{../images/圖5.6.png}
\caption{圖5.6}
\end{figure}

將所需零件加入後,這些零組件會自動出現在剛才提到的產品中。

\begin{figure}[h]
\centering
\includegraphics[width=0.8\textwidth]{../images/圖5.7.png}
\caption{圖5.7}
\end{figure}

將產品設定完後回到主頁面並選取新產品介紹下方的工程變更。

\begin{figure}[h]
\centering
\includegraphics[width=0.8\textwidth]{../images/圖5.8.png}
\caption{圖5.8}
\end{figure}

在這個頁面中我們選擇要製作的產品和物料清單,並且指派工作給各單位組員也可以設定完成期限或留下備註。

\begin{figure}[h]
\centering
\includegraphics[width=0.8\textwidth]{../images/圖5.9.png}
\caption{圖5.9}
\end{figure}

當我們設定完成後製作鋼球平衡台這項任務就會出現在頁面上。

\begin{figure}[h]
\centering
\includegraphics[width=0.8\textwidth]{../images/圖5.10.png}
\caption{圖5.10}
\end{figure}

團隊中的主管可以藉由拖曳將圖塊任務移到相對應的狀態底下,假如專案已完成,主管可將圖塊移到已完成區域,這些狀態可依情形不同做修改或增加。

\begin{figure}[h]
\centering
\includegraphics[width=0.8\textwidth]{../images/圖5.11.png}
\caption{圖5.11}
\end{figure}

若想更改用料清單可以使用主頁面中用料清單(BOM)更新的功能,使用方法與建立新產品雷同此處就不多贅述。

\begin{figure}[h]
\centering
\includegraphics[width=0.8\textwidth]{../images/圖5.12.png}
\caption{圖5.12}
\end{figure}

\begin{figure}[h]
\centering
\includegraphics[width=0.8\textwidth]{../images/圖5.13.png}
\caption{圖5.13}
\end{figure}