\chapter{背景與動機}
\renewcommand{\baselinestretch}{10.0} %設定行距
%\pagenumbering{arabic} %設定頁號阿拉伯數字




\fontsize{14pt}{2.5pt}\sectionef\hspace{12pt}
良好的團隊協同是製作一項專案的基本條件,GOOGLEMEET、MICROSOFT的TEAMS都是為了協同的需求所誕生的軟體,一項產品的發想、設計、製造各項步驟都需要良好的團隊配合才能將產品順利製造出來。\\[14pt]

\fontsize{14pt}{2.5pt}\sectionef\hspace{12pt} 在專案製作過程中團隊的資訊共享是十分重要的,而版本是協同設計的核心重點,我們利用ODDO中產品生命週期的功能,紀錄產品製造過程中的每一步歷程,這樣在專案製作過程中如果出現差錯,也能夠快速找出錯誤點並解決,之後若是有出現類似的專案也能夠藉由修改設計條件來得到產品,我們希望藉由專題展現ODDO在團隊協同上的應用並且實際製造一項產品。\\[12pt]


\newpage

\renewcommand{\baselinestretch}{0.5} %設定行距