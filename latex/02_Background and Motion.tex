\chapter{背景與動機}
\renewcommand{\baselinestretch}{10.0} %設定行距
%\pagenumbering{arabic} %設定頁號阿拉伯數字




\fontsize{14pt}{2.5pt}\sectionef\hspace{12pt}


專案製作過程中團隊的資訊共享至關重要,而版本控制則是協同設計的核心。我們借助ODDO中的產品生命週期管理功能,記錄產品製造過程中的每個步驟。如果在專案製作中出現錯誤,我們能夠快速定位問題並解決。我們希望透過實際製作一項產品,展現ODDO PLM在團隊協同上的應用。\\

 製造業企業在當今競爭激烈的市場環境中面臨著各種挑戰,包括產品生命週期管理(PLM)、協同設計和生產流程優化等。隨著全球化和數位化的發展,企業需要更有效的方法來管理產品開發過程,以滿足客戶需求並保持競爭力。\\



之所以選擇鋼球平衡台,是因為其結合自動控制、機構學、計算機概論、電腦輔助設計(CAD)、電腦輔助製造(CAM)等課程所學知識,並且具有相當的複雜度,能夠展現團隊的專業能力和協同合作的能力。\\

\newpage

\renewcommand{\baselinestretch}{0.5} %設定行距