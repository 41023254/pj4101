\chapter{結論}
\renewcommand{\baselinestretch}{10.0} %設定行距
%\pagenumbering{arabic} %設定頁號阿拉伯數字


\fontsize{14pt}{2.5pt}\sectionef\hspace{12pt}


在本次的專題中,我們使用Odoo作為產品設計的基底,並且結合SolidWorks 和 Coppeliasim設計零件及模擬,
再加入github進行協同,最後使用3D列印列印零件加以組裝然後配合機電控制實現從零開始生產並且製造出一個產品。在這個專裡我們學到了每種程式不同功用,
並使用各個程式的優缺點加以揉合後產生一加一大於二的功效。\\



在專題中我們在設計、仿真、協同、管理方面充分利用不同軟體平台的功能。首先我們使用 SolidWorks 來設計產品的尺寸及3D模型,
導入 Coppeliasim 中進行組合後的仿真和其運動的虛擬測試,以確保產品能正常運動和使用。同時在 Odoo 中的產品生命週期管理(PLM)模組創建相對應的管理系統,
以管理產品的生產、庫存和銷售,然後將過程在github裡面建立倉儲與其分支進行協同,最後3D列印出整個零件加入機電控制。這要求我們要具備跨越不同軟體平台的技術能力和整合能力,
以確保各個方面的協調和一致性。\\


有了這些便捷的軟體後我們可以先在SolidWorks 進行零件的初次設計,然後到Coppeliasim 中進行確實的仿真模擬。
我們可以在Coppeliasim中模擬產品在不同條件下的行為和性能,並進行虛擬測試及後續的改進及優化。這使我們能夠更好地了解產品的現實狀況及日後所需改進的問題,並更有效地設計和製造產品。\\


在專題中,我們認為ODOO不僅能定義產品的屬性還能根據產品的類型選定不同的模組進行使用,甚至還可以設置生產的流程,
物料清單及生產的計畫等等,其中最厲害的莫過於產品生命週期管理(PLM)模組,能從需求到設計開發到產品測試到大量生產到產品維護再到產品停產下架完成一整套的產品週期流程,
可謂是從零到有甚至於再到產品的終結。\\


整個專題中我們學到了如何獨當一面的設計和如何與其他成員協同分工,
在這個過程中不僅學到了許多新的技能和知識,更體驗到了團隊合作和創新思維的重要性,這些在未來職場上班時都將會寶貴的經驗。\\




\newpage

\renewcommand{\baselinestretch}{0.5} %設定行距