\renewcommand{\baselinestretch}{1.5} %設定行距
\pagenumbering{roman} %設定頁數為羅馬數字
\clearpage  %設定頁數開始編譯
\sectionef
\addcontentsline{toc}{chapter}{摘~~~要} %將摘要加入目錄
\begin{center}
\LARGE\textbf{摘~~~要}\\
\end{center}

\begin{flushleft}
\raggedright
\fontsize{14pt}{20pt}\sectionef\hspace{12pt}\quad 本專題主要研究有限元素法(FEM),由於近代計算機快速的發展,數值計算、開發環境、生程式設計等,都有公司或個人創作者製作軟體進行分析、計算,藉由這些軟體我們將對四足機器人進行生成式設計並且觀察其受力情況。\\[14pt]
\fontsize{14pt}{20pt}\sectionef\hspace{12pt}\quad 以四足機器人為例,將結構以剛體狀況導入CoppeliaSim進行動作模擬後,求出最大反力分別帶入Ansys和Solid Edge,並在此轉換為柔性結構,進行有限元素(FEM)分析,評估各柔性結構下分析的應力、應變等受力情況,對其做生成式設計以簡化模組,在保有強度的同時減輕重量造成最少的能源浪費。並嘗試透過網路展示CoppeliaSim機器人運動情況,證明其設計可行性。\\[12pt]

\end{flushleft}
\begin{center}
\vspace{6cm}
\fontsize{14pt}{20pt}\selectfont 關鍵字:偏微分方程(PDE)、有限元素分析(PEM)、CoppeliaSim、Ansys、Solid Edge
\end{center}
\newpage

%=--------------------Abstract----------------------=%
\renewcommand{\baselinestretch}{1.5} %設定行距
\addcontentsline{toc}{chapter}{Abstract} %將摘要加入目錄
\begin{center}
\LARGE\textbf\sectionef{Abstract}\\
\begin{flushleft}
\fontsize{14pt}{16pt}\sectionef\hspace{12pt}\quad The main focus of this project is on the Finite Element Method (FEM). With the rapid development of modern computers, numerical calculations, development environments, and software programming, various companies or individual creators have developed software for analysis and calculations. With the help of these software programs, we will perform generative design on a quadruped robot and observe its structural integrity under various load conditions.\\[12pt]

\fontsize{14pt}{16pt}\sectionef\hspace{12pt}\quad Taking the quadruped robot as an example, we will import the structure as a rigid body into CoppeliaSim for motion simulation. After obtaining the maximum reaction forces, we will input them into Ansys and Solid Edge for further analysis. The rigid structure will then be converted into a flexible structure to perform Finite Element Method (FEM) analysis. We will evaluate the stress, strain, and other load conditions for each flexible structure to assess their performance. Through generative design, we aim to simplify the modules while maintaining their strength and minimizing energy waste caused by excessive weight. Additionally, we will attempt to showcase the motion of the robot in CoppeliaSim through online demonstrations to prove the feasibility of the design.\\
\end{flushleft}
\begin{center}
\vspace{3cm}
\fontsize{14pt}{16pt}\selectfont\sectionef Keywords: partial differential equation (PDE), finite element analysis (PEM), CoppeliaSim, Ansys, Solid Edge.
\end{center}
